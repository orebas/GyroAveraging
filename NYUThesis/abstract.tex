%% Write your abstract here.
%

The gyroaverage operator, or 2d spherical mean transform, transforms a bivariate function by averaging values over circles of (potentially large) radius.  This operator has applications in plasma simulation as well as medical imaging, and its naive calculation for large radii is not well-tuned for modern, cache-sensitive processors.  We implement and benchmark a variety of schemes for calculating gyroaverages in the 2D, compactly supported non-periodic setting. Schemes implemented include bilinear spline collocation, bicubic spline collocation, padded (bivariate) FFT interpolation, and bivariate tensor Chebyshev interpolation. In particular we quantify the impact of shared-memory parallelism and the use of GPU accelerators to speed up calculations, as well as the trade-off of accuracy vs computational cost, for both smooth functions and those with singularities.