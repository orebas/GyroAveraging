\chapter{Background\label{chap:background}}

We use this chapter to record some basic facts we will use later.  Mainly we will be quoting from references.

\section{Harmonic Analysis\label{sec:Harmonic}}
%reference Zygmund, Boyd Spectral methods, Trefethen spectral methods
The below follows the notation and presentation from the expanded edition of \cite{trefethenATAP}, which includes the appendix ``ATAP for Periodic Functions."  Let $f$ be a Lipschitz continuous periodic function defined on $[0,2 \pi]$.  Then $f$ is associated to a Fourier series
\[  f(t) = \sum_{k=-\infty}^{k=\infty} c_k e^{ikt}\]
where the coefficients are given by the formula
\[c_k = \frac{1}{2\pi}  \int_{0}^{2\pi} f(t) e^{-ikt} \dd{t}. \]
We can truncate this expansion:
\[f_n(t) = \sum_{k=-n}^{n} c_k e^{ikt}\]
which is an approximation to $f$, the ``degree $n$ trigonometric projection" of $n$.\\
In practice, we may not have access to the exact coefficients $c_k$, but must rely only on values of $f$ sampled at $N$ points, generally equispaced in $[0,2\pi]$. Let $f_j = f \left( \frac{2 \pi j  }{N} \right)$.  Then we can use the formula
\[  \tilde{c}_k = \frac{1}{N} \sum_{j=0}^{N-1} f_k e^{-ikt}\] 
 which gives, via the same truncated expansion as above, the ``trigonometric interpolant" $p_n$ to $f$.   \\
 
 Now, if $f$ satisfies some smoothness condition, a standard argument involving repeated integration by parts yields the below error bounds.  This is Theorem B.2 in \cite[Appendix B]{trefethenATAP}.
 
 \newtheorem*{ATAPB-2}{Theorem B.2} 
 \begin{ATAPB-2}If $f$ is $\nu \geq 1$ times differentiable and $f^{(\nu)}$ is of bounded 
 variation $V$ on $[0,2\pi]$, then its degree $n$ trigonometric projection and interpolant satisfy
 \[ \norm{f - f_n}_\infty \leq \frac{V}{\pi \nu n^\nu}, \quad \norm{f - p_n}_\infty \leq \frac{2V}{\pi \nu n^\nu} \]
 If $f$ is analytic with $\abs{f(t) \leq M}$ in the open strip of half-width $\alpha$ around the real axis in the complex $t$-plane, they satisfy
 \[ \norm{f - f_n}_\infty \leq \frac{2Me^{-\alpha n}}{e^\alpha-1}, \quad \norm{f - p_n}_\infty \leq \frac{4M e^{-\alpha n}}{e^\alpha -1} \]
 \end{ATAPB-2}

\section{Harmonic Analysis of the Spherical Mean \label{sec:Harmonic2}}
Our interest in Fourier analysis is justified by the following calculation, which shows how the gyroaverage is greatly simplified in frequency space.  We focus on the bivariate case.  Suppose we have a Lipschitz continuous function $f$ on $\Omega = [0,2\pi]^2$, which is continuous and doubly periodic.  We have the Fourier series
\[ f(x,y) = \sum_{k=-\infty}^{\infty} \sum_{\ell = -\infty}^{\infty} c_{k,\ell} e^{2 \pi i (kx+\ell y)}  \]
where
\[ c_{k,\ell} = \int_{0}^{2 \pi} \int_{0}^{2 \pi} f(t_1,t_2) e^{-2 \pi i (kt_1 + \ell t_2)} \dd{t_1} \dd{t_2}\]

If we recall the definition of the gyroaverage
\[ \mathcal{G}(f)(x,y;\rho) = \frac{1}{2 \pi}\int_{0}^{2\pi} f(x+\rho \sin(\gamma), y + \rho \cos(\gamma)) \dd{\gamma}\]
We can ask what happens to a Fourier basis function under this operator.  Let
\[ f_{k, \ell} (x,y) = e^{2 \pi i (kx+\ell y)}   \]
Then
\[ \mathcal{G}(f_{k,\ell}) = \frac{1}{2 \pi}  \int_{0}^{2\pi} f_{k,\ell}(x+\rho \sin(\gamma), y + \rho \cos(\gamma)) \dd{\gamma} =  \]
\[  \frac{1}{2 \pi}  \int_{0}^{2\pi} e^{2 \pi i (k(x + \rho \sin \gamma) + \ell(y + \rho \cos \gamma))} \dd{\gamma} =   \frac{e^{2 \pi i (kx + \ell y)}}{2 \pi}\int_{0}^{2\pi} e^{2 \pi i \rho (k \sin \gamma + \ell \cos \gamma)} \dd{\gamma} =   \]
\[ f_{k,\ell} \frac{1}{2 \pi}\int_{0}^{2\pi} e^{2 \pi i \rho (k \sin \gamma + \ell \cos \gamma)} \dd{\gamma} \]
Next we use the identity $k \sin \gamma + \ell \cos \gamma = C \cos (\gamma + \phi)$, where $C = \sqrt{k^2 + \ell^2}$ and $\phi = \tan[-1](-\frac{k}{l})$.  The above is then 
\[ \mathcal{G}(f_{k,\ell}) = f_{k,\ell} \frac{1}{2 \pi} \int_{0}^{2 \pi} e^{2 \pi i \rho C \cos(\gamma + \phi)} = f_{k,l}  \cdot J_0(2 \pi \rho \sqrt{k^2 + \ell^2}) \]
Where $J_0(\cdot)$ is a Bessel function of the first kind, and the above is one form of the ``integral representation" found, e.g. in \cite{wikipediaBesselFunction} or \cite[Eq ~10.9.17]{NIST:DLMF}. 
%or abramowitz/stegun).  
Applied to the Fourier series expansion of an arbitrary function $f$, we see that
\[\mathcal{G}(f)(x,y;\rho) = \sum_{k=\infty}^{\infty} \sum_{\ell = \infty}^{\infty}  J_0(2 \pi \rho \sqrt{k^2 + \ell^2}) c_{k,\ell} e^{2 \pi i (kx+\ell y)}   \] 
This is expected, as the gyroaverage is a form of convolution; in Fourier space the operator is computed simply by a multiplier that depends only on $(\rho$, $k$, $l)$.  This fact is true, with similar derivation and slightly different Bessel functions, for all dimensions and for continuous Fourier transforms as well.

%\[ \frac{e^{2 \pi i (kx + \ell y)}}{2 \pi}
%\int_{0}^{2\pi} \cos(2 \pi \rho k \sin \gamma + 2 \pi \rho \ell \cos \gamma) + i \sin(2 \pi %\rho k \sin \gamma + 2 \pi \rho \ell \cos \gamma) \dd{\gamma}
% \]





\section{Chebyshev Approximation\label{Chebyshev}}
A full review of Chebyshev polynomials and their use in approximation theory is beyond the scope of this paper, and the primary subject of \cite{trefethenATAP}.  For one who is comfortable with Fourier analysis, the easiest point of view is that to approximate a function $f$ on $[-1,1]$, wrap it around the unit circle by the transform
\[ \theta = \cos^{-1} x \]
Then $f(\cos(\theta))$ is a periodic, even function  of $\theta$ which is then subject to decomposition and approximation by (cosine) Fourier methods.  We quote 3 theorems, all from \cite{trefethenATAP} which are analogous to the results for Fourier series above. \\

Define the Chebyshev polynomials as 
\[  T_k(x) = \cos(k \cos^{-1}(x))\]
Then any Lipschitz continuous function on $[-1,1]$ has a representation 
\[ f(x) = \sum_{k=0}^{\infty} a_k T_k(x) \]
with
\[ a_k = \frac{2}{\pi} \int_{-1}^{1} \frac{f(x) T_k(x)}{\sqrt{1-x^2}} \dd{x} \]
except that $\frac{2}{\pi}$ becomes $\frac{1}{\pi}$ for $k=0$.  
We can truncate this series and obtain 
\[ f_n(x) = \sum_{k=0}^{n} a_k T_k(x) \]
or, in analogy to the Fourier case, we can admit that in practice we may not have access to the integrals defining $a_k$.  Then we sample at Chebyshev nodes, $x_j = \cos \left(  \frac{j \pi}{n} \right)$ and interpolate to find coefficients $c_k$, such that\[p_n(x) =   \sum_{k=0}^{n} c_k T_k(x) \]
satisfies $p_n(x_j) = f(x_j)$.  (This interpolation is a discrete cosine transform, and is handled efficiently by FFT-based algorithms.)  Then we have the following theorems bounding the error of the above, quoted from \cite{trefethenATAP}.  We will not pause to define every term, and indeed Bernstein ellipses are somewhat technical.  The moral is that smoother functions will have faster convergence from interpolants than less smooth functions.
\newtheorem*{ATAP7-2}{Theorem 7.2 Convergence for differentiable functions}
\begin{ATAP7-2}
	For any integer $\nu \ge 0$, let $f$ and its derivatives through $f^{\nu -1}$ be absolutely continuous on $[-1,1]$ and suppose the $\nu$th derivative is of bounded variation $V$, then for any $n > \nu$, the Chebyshev projections satisfy 
	\[  \norm{f - f_n} \leq \frac{2V}{\pi \nu (n - \nu)\nu} \]
	and its Chebyshev interpolants satisfy
	\[ \norm {f- p_n} \leq \frac{4V}{\pi \nu (n-\nu)^{\nu}}\]
\end{ATAP7-2}
\newtheorem*{ATAP8-2}{Theorem 8.2 Convergence for analytic functions}
\begin{ATAP8-2}
	Let a function $f$ analytic in $[-1,1]$ be analytically continuable to the open Bernstein ellipse $E_\rho$, where it satisfies $\abs{f(x)}\leq M$ for some $M$.  Then for each $n \geq 0$ its Chebyshev projections satisfy 
	\[  \norm{f - f_n} \leq \frac{2 M \rho^{-n} }{\rho - 1} \]
	and its Chebyshev interpolants satisfy
	\[ \norm {f- p_n} \leq \frac{4 M \rho^{-n} }{\rho - 1}\]
\end{ATAP8-2}



%figure~\ref{fig:afigure}.
%\begin{figure}[htb]
%  \begin{center}
%    \emph{This statement is false.}
%  \end{center}%
%\caption[This alternate caption appears in the list of figures.]{This is the
%caption that appears under the figure. It may be quite long---you wouldn't want
%such a long caption to appear in the ``list of figures''.}
%\label{fig:afigure}
%\end{figure}

%More blah, blah. There is nothing interesting about
%table~\ref{tab:atable} either.
%\begin{table}[htb]
%\caption[Strange rules.]{For some reason unfamiliar to me,
%typesetting rules require one to place captions above tables,
%but below figures. Go figure.}\label{tab:atable}
%  \begin{center}
%    \framebox{You could put a table here. I won't.}
%  \end{center}
%\end{table}

